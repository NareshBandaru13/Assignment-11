\documentclass{beamer}
\usetheme{CambridgeUS}

\setbeamertemplate{caption}[numbered]{}

\usepackage{enumitem}
\usepackage{amsmath}
\usepackage{amssymb}
\usepackage{gensymb}
\usepackage{graphicx}
\usepackage{txfonts}

\def\inputGnumericTable{}

\usepackage[latin1]{inputenc}                                 
\usepackage{color}                                            
\usepackage{array}                                            
\usepackage{longtable}                                        
\usepackage{calc}                                             
\usepackage{multirow}                                         
\usepackage{hhline}                                           
\usepackage{ifthen}
\usepackage{caption}

\title{AI1110 \\ Assignment 11}
\author{Bandaru Naresh Kumar \\ AI21BTECH11006}
\date{}
\begin{document}
	% The title page
	\begin{frame}
		\titlepage
	\end{frame}
	
	% The table of contents
	\begin{frame}{Outline}
    		\tableofcontents
	\end{frame}
	
	% The question
	\section{Question}
	\begin{frame}{Exercise 15.12}
       In the genetic model in (15-31),consider the possibility that prior to the formation of a new generation each gene can spontaneously mutate into a gene of the other kind with probabilities\\
       $P(A\rightarrow B) = \alpha(>0)$ and\\
       $P(B\rightarrow A) = \beta(>0)$\\
       Thus for a system in state $e_j$, after mutation there are $N_A = j(1-\alpha)+(N-1)\beta$ genes of type A and $N_B = j\alpha = (N-1)(1-\beta)$ genes of type B. 
	\end{frame}
	
	\begin{frame}{Exercise 15.12}
	  Hence the modified probabilities prior to forming a new generation are\\
	  $p_j = \dfrac{N_A}{N} = \dfrac{j}{N}(1-\alpha)+(1-\dfrac{j}{N})\beta$ and\\
	  $q_j = \dfrac{N_B}{N} = \dfrac{1}{N}\alpha+(1-\dfrac{j}{N})(1-\beta)$\\
	  for the A and B genes,respectively.	
	\end{frame}
	
	\begin{frame}{Exercise 15.12}
	This gives\\
	$p_{jk} = \binom{N}{k} p_j^k q_j^{N-k}$ where j,k = 0,1,2,....,N\\
	to be the modified transition probabilities for the Morkov chain with mutation.Derive the steady state distribution for this model,and show that,unlike the models in (15-30) and (15-31),fixation to "the pure gene states" does not occur in this case.
	\end{frame}
	
	% The solution
	\section{Solution}
	\begin{frame}{Solution}
	     In this case,the chain is irreducible and aperiodic and there are no absorption states.\\
	     The steady state distribution $\{u_k\}$ satisfies:\\
	     $u_k = \sum_j u_jp_{jk} = \sum_{j=0}^N u_j \binom{N}{k} p_j^kq_j^{N-k}$\\
	     So if $\alpha > 0$ and $\beta > 0$,then "fixation to pure genes" does not occur.    	     
	 \end{frame}
	  
\end{document}